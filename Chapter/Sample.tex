\documentclass[10pt]{ietbook}

\begin{document}

%For RH Book title
\rhbooktitle{Book title}

\markboth{Running head verso book title}{Running head recto chapter title}

\cauthor{Author Name One\thanks{Author affiliations}
Author Name Two\thanks{Author affiliations} and\\
Author Name Three\thanks{Author affiliations}}

\chapter{Title text}

The architecture of the can be roughly sketched
as consisting of a bottom sensor layer, a middle network layer, and a
top application layer. As one of the primary information-acquiring means
at the bottom layer of the
tags have found increasingly widespread applications in various business
areas, with the expectation that the use of RFID tags will eventually
replace the existing bar codes in all business areas. 

The architecture of the can be roughly sketched
as consisting of a bottom sensor layer, a middle network layer, and a
top application layer. As one of the primary information-acquiring means
at the bottom layer of the
tags have found increasingly widespread applications in various business
areas, with the expectation that the use of RFID tags will eventually
replace the existing bar codes in all business areas. 
\index{fa series index text index!bcde index text index text index text!cdef index text index text} 
\index{fb series index text index text index!bcde index text index text!cdef index text}\index{fb series index text index text index!cdef index text index text index text}

\section{A heading (level 1)}

The architecture of the can be roughly sketched
as consisting of a bottom sensor layer, a middle network layer, and a
top application layer. As one of the primary information-acquiring means
at the bottom layer of the
tags have found increasingly widespread applications in various business
areas, with the expectation that the use of RFID tags will eventually
replace the existing bar codes in all business areas.\index{ga series index text index!bcde index text index text!cdef index text index text}

The architecture of the can be roughly sketched
as consisting of a bottom sensor layer, a middle network layer, and a
top application layer. As one of the primary information-acquiring means
at the bottom layer of the
tags have found increasingly widespread applications in various business
areas, with the expectation that the use of RFID tags will eventually
replace the existing bar codes in all business areas.\index{gb series index text index!bcde index text text!cdef index text text index text}

\subsection{B heading (level 2)}

The architecture of the can be roughly sketched
as consisting of a bottom sensor layer, a middle network layer, and a
top application layer. As one of the primary information-acquiring means
at the bottom layer of the
tags have found increasingly widespread applications in various business
areas, with the expectation that the use of RFID tags will eventually
replace the existing bar codes in all business areas.\index{ha series index text index!bcde index text index text index text!cdef index text index text}

The architecture of the can be roughly sketched
as consisting of a bottom sensor layer, a middle network layer, and a
top application layer. As one of the primary information-acquiring means
at the bottom layer of the
tags have found increasingly widespread applications in various business
areas, with the expectation that the use of RFID tags will eventually
replace the existing bar codes in all business areas.\index{hb series index text index!bcde index text index text index text!cdef index text index text}

\subsubsection{C heading (level 3)}

The architecture of the can be roughly sketched
as consisting of a bottom sensor layer, a middle network layer, and a
top application layer. As one of the primary information-acquiring means
at the bottom layer of the
tags have found increasingly widespread applications in various business
areas, with the expectation that the use of RFID tags will eventually
replace the existing bar codes in all business areas.\index{hc series index text index!bcde index text index text!cdef index text index }

The architecture of the can be roughly sketched
as consisting of a bottom sensor layer, a middle network layer, and a
top application layer. As one of the primary information-acquiring means
at the bottom layer of the
tags have found increasingly widespread applications in various business
areas, with the expectation that the use of RFID tags will eventually
replace the existing bar codes in all business areas.\index{ia series index text index index!bcde index text index text text!cdef index text index text}

\begin{figure}[!b]
\centerline{\fbox{\hbox to 20pc{\vbox to 10pc{}}}}
\caption{Caption text the architecture of the can be roughly sketched
as consisting of a bottom sensor layer, a middle network layer, and a
top application layer
at the bottom layer of the
tags have found increasingly widespread applications in various business
areas}
\end{figure}

\paragraph{D heading (level 4)}

The architecture of the can be roughly sketched
as consisting of a bottom sensor layer, a middle network layer, and a
top application layer. As one of the primary information-acquiring means
at the bottom layer of the
tags have found increasingly widespread applications in various business
areas, with the expectation that the use of RFID tags will eventually
replace the existing bar codes in all business areas.\index{ib series index text index text index!bcde index text index text!cdef index text index text}

The architecture of the can be roughly sketched
as consisting of a bottom sensor layer, a middle network layer, and a
top application layer. As one of the primary information-acquiring means
at the bottom layer of the
tags have found increasingly widespread applications in various business
areas, with the expectation that the use of RFID tags will eventually
replace the existing bar codes in all business areas.\index{ic series index text index text index!bcde index text!cdef index text index text index text}

The architecture of the can be roughly sketched
as consisting of a bottom sensor layer, a middle network layer, and a
top application layer. As one of the primary information-acquiring means
at the bottom layer of the
tags have found increasingly widespread applications in various business
areas, with the expectation that the use of RFID tags will eventually
replace the existing bar codes in all business areas.\index{ja series index text index!bcde index text index text!cdef index text index text index text}

\begin{figure}[!t]%
\centerline{\fbox{\hbox to 20pc{\vbox to 10pc{}}}}
\caption{Caption text}
\end{figure}


\begin{boxes}
{\boxhead{A heading}}
{The architecture of the can be roughly sketched
as consisting of a bottom sensor layer, a middle network layer, and a
top application layer. As one of the primary information-acquiring means
at the bottom layer of the
tags have found increasingly widespread applications in various business
areas, with the expectation that the use of RFID tags will eventually
replace the existing bar codes in all business areas.\index{jc series index text index text index!bcde index text!cdef index text index text index text}

The architecture of the can be roughly sketched
as consisting of a bottom sensor layer, a middle network layer, and a
top application layer. As one of the primary information-acquiring means
at the bottom layer of the
tags have found increasingly widespread applications in various business
areas, with the expectation that the use of RFID tags will eventually
replace the existing bar codes in all business areas.\index{ka series index text index text index!bcde index text!cdef index text index text index text}}
\end{boxes}

The architecture of the can be roughly sketched
as consisting of a bottom sensor layer, a middle network layer, and a
top application layer. 
As one of the primary information-acquiring means
at the bottom layer of the
tags have found increasingly widespread applications in various business
areas, with the expectation that the use of RFID tags will eventually
replace the existing bar codes in all business areas.\index{jb series index text index!bcde index text index text!cdef index text index text index text}

The architecture of the can be roughly sketched
as consisting of a bottom sensor layer, a middle network layer, and a
top application layer. As one of the primary information-acquiring means
at the bottom layer of the
tags have found increasingly widespread applications in various business
areas, with the expectation that the use of RFID tags will eventually
replace the existing bar codes in all business areas.\index{kb series index text index!bcde index text index text!cdef index text index text index text}

The architecture of the can be roughly sketched
as consisting of a bottom sensor layer, a middle network layer, and a
top application layer. As one of the primary information-acquiring means
at the bottom layer of the
tags have found increasingly widespread applications in various business
areas, with the expectation that the use of RFID tags will eventually
replace the existing bar codes in all business areas.\index{kc series index text index text index!bcde index text index text!cdef index text index text index text}
The architecture of the can be roughly sketched
as consisting of a bottom sensor layer, a middle network layer, and a
top application layer. As one of the primary information-acquiring means
at the bottom layer of the
tags have found increasingly widespread applications in various business
areas, with the expectation that the use of RFID tags will eventually
replace the existing bar codes in all business areas.\index{kd series index text index text index!bcde index text index text!cdef index text index text index text}

The architecture of the can be roughly sketched
as consisting of a bottom sensor layer, a middle network layer, and a
top application layer. As one of the primary information-acquiring means
at the bottom layer of the
tags have found increasingly widespread applications in various business
areas, with the expectation that the use of RFID tags will eventually
replace the existing bar codes in all business areas.\index{la series index text index text index!bcde index text index text index text!cdef index text index text}

The architecture of the can be roughly sketched
as consisting of a bottom sensor layer, a middle network layer, and a
top application layer. As one of the primary information-acquiring means
at the bottom layer of the
tags have found increasingly widespread applications in various business
areas, with the expectation that the use of RFID tags will eventually
replace the existing bar codes in all business areas.\index{lb series index text index text index!bcde index text index text index text!cdef index text index text index text}

The architecture of the can be roughly sketched
as consisting of a bottom sensor layer, a middle network layer, and a
top application layer. As one of the primary information-acquiring means
at the bottom layer of the
tags have found increasingly widespread applications in various business
areas, with the expectation that the use of RFID tags will eventually
replace the existing bar codes in all business areas.\index{lc series index text index!bcde index text index text index text!cdef index text index text index text}

The architecture of the can be roughly sketched
as consisting of a bottom sensor layer, a middle network layer, and a
top application layer. As one of the primary information-acquiring means
at the bottom layer of the
tags have found increasingly widespread applications in various business
areas, with the expectation that the use of RFID tags will eventually
replace the existing bar codes in all business areas.\index{ld series index text index text index!bcde index text index text index text!cdef index text index text}

The architecture of the can be roughly sketched
as consisting of a bottom sensor layer, a middle network layer, and a
top application layer. As one of the primary information-acquiring means
at the bottom layer of the
tags have found increasingly widespread applications in various business
areas, with the expectation that the use of RFID tags will eventually
replace the existing bar codes in all business areas.\index{ma series index text index text index!bcde index text index text!cdef index text index text index text}

The architecture of the can be roughly sketched
as consisting of a bottom sensor layer, a middle network layer, and a
top application layer. As one of the primary information-acquiring means
at the bottom layer of the
tags have found increasingly widespread applications in various business
areas, with the expectation that the use of RFID tags will eventually
replace the existing bar codes in all business areas.\index{mb series index text index text index!bcde index text index text!cdef index text index text index text}

The architecture of the can be roughly sketched
as consisting of a bottom sensor layer, a middle network layer, and a
top application layer. As one of the primary information-acquiring means
at the bottom layer of the
tags have found increasingly widespread applications in various business
areas, with the expectation that the use of RFID tags will eventually
replace the existing bar codes in all business areas.\index{na series index text index text index!bcde index text!cdef index text index text index text}

The architecture of the can be roughly sketched
as consisting of a bottom sensor layer, a middle network layer, and a
top application layer. As one of the primary information-acquiring means
at the bottom layer of the
tags have found increasingly widespread applications in various business
areas, with the expectation that the use of RFID tags will eventually
replace the existing bar codes in all business areas.\index{nb series index text index!bcde index text!cdef index text index text}
The architecture of the can be roughly sketched
as consisting of a bottom sensor layer, a middle network layer, and a
top application layer. As one of the primary information-acquiring means
at the bottom layer of the
tags have found increasingly widespread applications in various business
areas, with the expectation that the use of RFID tags will eventually
replace the existing bar codes in all business areas.\index{oa series index!bcde index text index text!cdef index text index text index text}

The architecture of the can be roughly sketched
as consisting of a bottom sensor layer, a middle network layer, and a
top application layer. As one of the primary information-acquiring means
at the bottom layer of the
tags have found increasingly widespread applications in various business
areas, with the expectation that the use of RFID tags will eventually
replace the existing bar codes in all business areas.\index{ob series index text index!bcde index text index text!cdef index text index text}

%Table 
\begin{table}[!b]
\processtable{Table title the architecture of the can be roughly sketched
consisting of a bottom sensor layer, a middle network layer, and a top application~layer}
{\begin{tabular*}{\textwidth}{@{\extracolsep{\fill}}lllll@{}}\toprule
\TCH{Column} & \TCH{Column heads} & \TCH{Column} & \TCH{Column} & \TCH{Column} \\
\TCH{heads} &  & \TCH{heads} & \TCH{heads} & \TCH{heads} \\\midrule
Body text & Body of the text & Body text & Body text & Body text \\
Body text & Body of the text & Body text & Body text & Body text \\
Body text & Body of the text & Body text & Body text & Body text \\
Body text & Body of the text & Body text & Body text & Body text \\
Body text & Body of the text & Body text & Body text & Body text \\
Body text$^{1}$ & Body text$^{2}$ & Body text & Body text & Body text \\\botrule
\end{tabular*}}{$^{1}$Table footnote text\\
$^{2}$Table footnote text\\
\textit{Source}: text roughly sketched.}
\end{table}

The architecture of the can be roughly sketched
as consisting of a bottom sensor layer, a middle network layer, and a
top application layer. As one of the primary information-acquiring means
at the bottom layer of the
tags have found increasingly widespread applications in various business
areas, with the expectation that the use of RFID tags will eventually
replace the existing bar codes in all business areas.\index{pa series index text index!bcde index text index text!cdef index text index text index text index text}

As one of the primary information-acquiring means
at the bottom layer of the
tags have found increasingly widespread applications in various business
areas, with the expectation that the use of RFID tags will eventually
replace the existing bar codes in all business areas.
\begin{equation}
\alpha  =  \left(\frac{1_{x}}{2_{z}} \right)
\end{equation}
The architecture of the can be roughly sketched
as consisting of a bottom sensor layer, a middle network layer, and a
top application layer. 
\[
\beta = \left( \frac{1}{1 + \alpha} \right)
\]
The architecture of the can be roughly sketched
as consisting of a bottom sensor layer, a middle network layer, and a
top application layer.\index{pb series index text index!bcde index text index text!cdef index text index text index text}

%Table 
\begin{table}[!t]
\processtable{Table title text increasingly widespread applications}
{\begin{tabular*}{15pc}{@{\extracolsep{\fill}}lll@{}}\toprule
\TCH{Column heads} & \TCH{Column heads} & \TCH{Column} \\
 &  & \TCH{heads}  \\\midrule
Body of the text & Body of the text & Body text \\
Body of the text & Body of the text & Body text \\
Body of the text$^{1}$ & Body text$^{2}$ & Body text \\\botrule
\end{tabular*}}{$^{1}$Table footnote text}
\end{table}

The architecture of the can be roughly sketched
as consisting of a bottom sensor layer, a middle network layer, and a
top application layer. 
\begin{theorem}[Theorem subhead]
{As one of the primary information-acquiring means at the bottom layer of the
tags have found increasingly widespread applications in various business
areas, with the expectation that the use of RFID tags will eventually
replace the existing bar codes in all business areas.}
\end{theorem}
The architecture of the can be roughly sketched
as consisting of a bottom sensor layer, a middle network layer, and a
top application layer. As one of the primary information-acquiring means
at the bottom layer of the
tags have found increasingly widespread applications in various business
areas, with the expectation that the use of RFID tags will eventually
replace the existing bar codes in all business areas.\index{qa series index text index!bcde index text index text!cdef index text index text index text}

The architecture of the can be roughly sketched
as consisting of a bottom sensor layer, a middle network layer, and a
top application layer. \index{ra series index text index text index!bcde index text index text index text!cdef index text index text index text}
\begin{lemma}[Lemma subhead]
{As one of the primary information-acquiring means
at the bottom layer of the
tags have found increasingly widespread applications in various business
areas, with the expectation that the use of RFID tags will eventually
replace the existing bar codes in all business areas.}
\end{lemma}
The architecture of the can be roughly sketched
as consisting of a bottom sensor layer, a middle network layer, and a
top application layer.\index{sa series index text index text index!bcde index text index text index text!cdef index text}

The architecture of the can be roughly sketched
as consisting of a bottom sensor layer, a middle network layer, and a
top application layer. 
\begin{proposition}[Proposition subhead]
{As one of the primary information-acquiring means
at the bottom layer of the
tags have found increasingly widespread applications in various business
areas, with the expectation that the use of RFID tags will eventually
replace the existing bar codes in all business areas.}
\end{proposition}
The architecture of the can be roughly sketched
as consisting of a bottom sensor layer, a middle network layer, and a
top application layer.\index{ta series index text index text index!bcde index text index text index text!cdef index text}

The architecture of the can be roughly sketched
as consisting of a bottom sensor layer, a middle network layer, and a
top application layer. 
\begin{remark}[Remark subhead]
{As one of the primary information-acquiring means
at the bottom layer of the
tags have found increasingly widespread applications in various business
areas, with the expectation that the use of RFID tags will eventually
replace the existing bar codes in all business areas.}
\end{remark}
The architecture of the can be roughly sketched
as consisting of a bottom sensor layer, a middle network layer, and a
top application layer.\index{ua series index text index text index!bcde index text index text index text!cdef index text index text}

The architecture of the can be roughly sketched
as consisting of a bottom sensor layer, a middle network layer, and a
top application layer. 
\begin{property}[Propery subhead]
{As one of the primary information-acquiring means at the bottom layer of the
tags have found increasingly widespread applications in various business
areas, with the expectation that the use of RFID tags will eventually
replace the existing bar codes in all business areas.}
\end{property}
The architecture of the can be roughly sketched
as consisting of a bottom sensor layer, a middle network layer, and a
top application layer.\index{va series index text index text index!bcde index text index text index text!cdef index text index text}


The architecture of the can be roughly sketched
as consisting of a bottom sensor layer, a middle network layer, and a
top application layer.\index{wa series index text index!bcde index text index text index text!cdef index text index text}
\begin{example}[Example subhead]
{As one of the primary information-acquiring means at the bottom layer of the
tags have found increasingly widespread applications in various business
areas, with the expectation that the use of RFID tags will eventually
replace the existing bar codes in all business areas.}
\end{example}
The architecture of the can be roughly sketched
as consisting of a bottom sensor layer, a middle network layer, and a
top application layer.\index{eb series index text index text index!bcde index text index text index text!cdef index text index text}\index{xa series index text index!bcde index text index text!cdef index text index text}

As one of the primary information-acquiring means at the bottom layer of the
tags have found increasingly widespread applications in various business
areas, with the expectation that the use of RFID tags will eventually
replace the existing bar codes in all business areas.
\index{ya series index!bcde index text!cdef index text index text}
\begin{definition}
The architecture of the can be roughly sketched
as consisting of a bottom sensor layer, a middle network layer, and a
top application layer.
\end{definition}
As one of the primary information-acquiring means at the bottom layer of the
tags have found increasingly widespread applications in various business
areas, with the expectation that the use of RFID tags will eventually
replace the existing bar codes in all business areas.\index{ea series index text}\index{za series index text index text index!bcde index text index text index text!cdef index text index text index text index text index text}

The architecture of the can be roughly sketched
as consisting of a bottom sensor layer, a middle network layer, and a
top application layer.\index{dd series index text index text index!abcd index text index text index text!abcd index text index text index text index text index text}
\begin{description}
\item [Description bold texts:] The architecture of the can be roughly sketched as consisting of a bottom sensor layer, a middle network layer, and a top application layer. 
\item [Description bold texts:] As one of the primary information-acquiring means at the bottom layer of the tags have found increasingly widespread applications in various business areas.
\item [Description bold texts:] The architecture of the can be roughly sketched as consisting of a bottom sensor layer, a middle network layer, and a top application layer.\index{ec series index text}
\end{description}
Between these two approaches,
it has been realized that there are essentially no feasible and
effective ways to completely prevent an RFID tag from being cloned.
Hence, much effort have been devoted to the cloned tag detection
techniques.\index{eb series index text!abcd index text!efgh index text}

Between these two approaches,
it has been realized that there are essentially no feasible and
effective ways to completely prevent an RFID tag from being cloned.
Hence, much effort have been devoted to the cloned tag detection
techniques. 
\begin{enumerate}
\item{} Numbered List style text\index{eb series index text!abcd
index text!efgh index text} between these two approaches,
it has been realized that there are essentially no feasible and
effective ways to completely prevent an RFID tag from being cloned.
Hence, much effort have been devoted to the cloned tag detection
techniques. 
\item{} Numbered List style text\index{ec series index text} between these two approaches,
it has been realized that there are essentially no feasible and
effective ways to completely prevent an RFID tag from being cloned.
Hence, much effort have been devoted to the cloned tag detection
techniques. 
\item{} Numbered List style text between these two approaches,
it has been realized that there are essentially no feasible and
effective ways to completely prevent an RFID tag from being cloned.
Hence, much effort have been devoted to the cloned tag detection
techniques. 
\end{enumerate}
Between these two approaches, it has been realized that there are essentially no feasible and
effective ways to completely prevent an RFID tag from being cloned.
Hence, much effort have been devoted to the cloned tag detection
techniques. 

Between these two approaches, it has been realized that there are essentially no feasible and
effective ways to completely prevent an RFID tag from being cloned.
Hence, much effort have been devoted to the cloned tag detection techniques. 
\begin{unnumlist}
\item{} Unnumlist style text between these two approaches, it has been realized that there are essentially no feasible and effective ways to completely prevent an RFID tag from being cloned. 
\item{} Unnumlist style text hence, much effort have been devoted to the cloned tag detection techniques.
\item{} Unnumlist style text between these two approaches, it has been realized that there are essentially no feasible and effective ways to completely prevent an RFID tag from being cloned. 
\end{unnumlist}
Between these two approaches, it has been realized that there are essentially no feasible and
effective ways to completely prevent an RFID tag from being cloned.
Hence, much effort have been devoted to the cloned tag detection techniques.\index{da series index text index text index!abcd index text index text
index text!abcd index text index text index text index text index text}

Between these two approaches, it has been realized that there are essentially no feasible and
effective ways to completely prevent an RFID tag from being cloned.
Hence, much effort have been devoted to the cloned tag detection techniques.\index{d series index text}

Between these two approaches, it has been realized that there are essentially no feasible and
effective ways to completely prevent an RFID tag from being cloned.
Hence, much effort have been devoted to the cloned tag detection techniques.\index{db series index text index text index!bcde index text index text index text!cdef index text index text index text index text index text}
\begin{itemize}
\item{} For instance, a box of goods may be split into sub-boxes of goods at a distribution center, and several (different types of) products may be bundled together for sale at a wholesale store.
\item{} One notable method of detecting cloned tags is to check the history record of events of the suspected tag against some predefined criteria.
\item{} For instance, a box of goods may be split into sub-boxes of goods at a distribution center, and several (different types of) products may be bundled together for sale at a wholesale store.
\end{itemize}
For instance, a box of goods may be split into sub-boxes of goods at a distribution center, and several (different types of) products may be bundled together for sale at a wholesale store.\footnote{One notable method of detecting cloned RFID tags is to check the history record of events of the suspected tag against some predefined criteria.}


One notable method of detecting cloned:
\begin{quote}
extract style text the primary means to prevent cloned RFID tags from being made is to
encrypt the data contained in RFID tags via public/private keys to make
it difficult to be revealed and cloned, and many such mechanisms have
been proposed.

extract style text the primary means to prevent cloned RFID tags from being made is to
encrypt the data contained in RFID tags via public/private keys to make
it difficult to be revealed and cloned, and many such mechanisms have
been proposed.

extract style text the primary means to prevent cloned RFID tags from being made is to
encrypt the data contained in RFID tags via public/private keys to make
it difficult to be revealed and cloned, and many such mechanisms have been proposed.
\end{quote}
the primary means to prevent cloned RFID tags from being made is to
encrypt the data contained in RFID tags via public/private keys to make
it difficult to be revealed and cloned, and many such mechanisms have
been proposed the primary means to prevent cloned RFID tags from being made is to
encrypt the data contained in RFID tags via public/private keys to make
it difficult to be revealed and cloned, and many such mechanisms have
been proposed the primary means to prevent cloned RFID tags from being made is to
encrypt the data contained in RFID tags via public/private keys to make
it difficult to be revealed and cloned, and many such mechanisms have
been proposed the primary means to prevent cloned RFID tags from being made is to
encrypt the data contained in RFID tags via public/private keys to make
it difficult to be revealed and cloned, and many such mechanisms have
been proposed.

The first protocol allows tags and readers/writers to authenticate to\cite{uniform} 
each other, and the second protocol only requires Hash or XOR operations\cite{bibliographic}
be performed in tag memory authentication protocol which not only\cite{halpern.ubel.ea:solid-organ*2} 
provides authentications but disguises tags against attackers by giving\cite{halpern.ubel.ea:solid-organ}
them fake names, as well.\cite{halpern.ubel.ea:solid-organ*1}

The first protocol allows tags and readers/writers to authenticate to\cite{rose.huerbin.ea:regulation}
each other, and the second protocol only requires Hash or XOR operations\cite{hypertension}
be performed in tag memory authentication protocol which not only\cite{vallancien.emberton.ea:sexual}
provides authentications but disguises tags against attackers by giving\cite{21st}
them fake names, as well.\cite{ellingsen.wilhelmsen:sykdomsangst}

\bibliographystyle{vancouver-modified}
\bibliography{sample-vancouver}

\end{document}
