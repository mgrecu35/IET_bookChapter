
\documentclass[10pt]{ietbook}
\usepackage{listings}
\begin{document}

%For RH Book title
\rhbooktitle{Book title}
%\setminted[python]{breaklines, framesep=2mm, fontsize=\footnotesize, numbersep=5pt}

\markboth{Running head verso book title}{Running head recto chapter title}

\cauthor{Mircea Grecu\thanks{GESTAR-II, Morgan State University and NASA GSFC} and\\
William S. Olson\thanks{GESTAR-II,UMBC and NASA GSFC} }

\chapter{Satellite Combined Radar-Radiometer algorithms}

Some abstract here.
\section{Introduction}

The benefit of incorporating radiometer observations into methodologies that estimate precipitation from
space-borne radar observations was first realized by \cite{weinman90}. Specifically, due to size and weight limitations on antennae, space-borne radars 
operate at frequencies that make observations subject to attenuation. While
attenuation correction methodologies exist, e.g. \cite{hitschfeld1954}, variability in the size distribution of
precipitation particles within the radar observing volume makes the attenuation correction process highly uncertain.
At the same time, it had been recognized \cite{Meneghini1983} that independent information regarding the Path-Integrated-Attenuation (PIA) 
may be used to reduce uncertainties in the attenuation correction process. PIA information
independent of the radar measurements used in the attenuation correction process may be derived from the analysis
of the electromagnetic power backscattered by the Earth's surface \cite{Meneghini1983} and, as noted by Weinman et al. \cite{weinman90}, 
low-frequency radiometer observations when available.  The objective of the surface analysis is to estimate the power backscattered by Earth's surface in the 
absence of the rain.  In the presence of rain, the ratio between the actual backscattered power and the estimated clear-sky backscattered power provides an estimate
of the total attenuation from the radar to the Earth's surface \cite{Meneghini1983} that can be used in attenuation correction and precipitation estimation process. 
The problem with the PIA estimation using this approach, usually referred to as the Surface Reference Technique (SRT), is that the estimation of the no-precipitation 
backscattered power may be highly uncertain in some situations.  As recognized by Weinman et al. \cite{weinman90} and even earlier investigators, e.g. \cite{Fujita1985},
radiometer observations over water surfaces at frequencies not associated with significant scattering (e.g. 10- and 19-GHz) contain information strongly related to the PIA.  
This is because the emissivity of water surface is low and rain drops in the radiometer observing volume result in warmer brightness temperatures.  The departure from the
brightness temperature value expected in clear skies may be used to estimate the equivalent PIA at the radar operating frequency \cite{weinman90}.  The initial study of
Weinman et al. \cite{weinman90} provided a relationship between coincident radiometer observations at X-band and the associated PIA at X-band (9.6-GHz). Subsequent work 
Smith et al. \cite{smith1997} was carried out to determine a relationship between the PIA at Ku-band (13.8 GHz) and coincident radiometer observations 10.7-GHz. 
This relationship was developed for use in the Tropical Rainfall Measuring Mission (TRMM) \cite{kummerow1998} combined radar-radiometer algorithm \cite{haddad1997}.

However, although the relationships between low frequency brightness temperatures and the radar PIA at X- and Ku-band are well-defined and unambiguous, the use of satellite
radiometer observations in satellite radar profiling algorithms is challenging. This because the typical footprints of low-frequency radiometer observations are significantly
larger than the typical footprints of space-borne radars, which makes the radiometer observations difficult to translate into radar PIA. To overcome this difficulty various
approaches akin to the downscaling of the radiometer observations to the radar footprint resolution have been developed \cite{haddad1997,grecu2004,masunaga2005,munchak2011}.
A common feature of these approaches was that the radar-footprint PIA was not directly estimated from the radiometer observations. Instead, 
optimization procedures were used to maximize the agreement between radiometer observations predicted from the radar observations and actual radiometer observations.  While
the impact of the radiometer observations on the final radar estimates is difficult to quantify, a clear benefit of combined radar-radiometer precipitation retrievals
are consistent with both the radar and radiometer observations.  Consequently, they may be used to derive large databases of precipitation and associated radiometer observations
necessary in the development of “Bayesian” precipitation estimation algorithms from satellite
radiometer-only observations \cite{grecu2006,kummerow2011,hou2014}. 


%\begin{table}[!tb]
%\caption{A table}
%\processtable{ Example of code }
\lstset{language=Python}
\lstset{frame=lines}
\lstset{caption={Insert code directly in your document}}
%\lstset{label={lst:code_direct}}
%\lstset{basicstyle=\footnotesize}
\begin{lstlisting}
from brg.datastructures import Mesh
 
mesh = Mesh.from_obj('faces.obj')
mesh.draw()
\end{lstlisting}
%\end{table}

%\subsection{B heading (level 2)}

%\subsubsection{C heading (level 3)}

%\begin{figure}[!b]
%\centerline{\fbox{\hbox to 20pc{\vbox to 10pc{}}}}
%\caption{little figure here}
%\end{figure}

%\paragraph{D heading (level 4)}



\bibliographystyle{vancouver-modified}
\bibliography{sample-vancouver}

\end{document}
