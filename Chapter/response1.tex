\documentclass[12pt]{article}
\usepackage{amsmath,amssymb,amsthm}
\usepackage{graphicx}
\author{Mircea Grecu and William S. Olson}
\title{Satellite Combined Radar-Radiometer
algorithms\\
Responses to Reviewer 1}
\date{}
\begin{document}
\maketitle


We thank the reviewer for the insightful comments and suggestions. 
Here are our point by point responses.

\begin{enumerate}
\item	{\textit{variability could be also due to Nw...in fact more so than Dm? The independent information from PIA...what is this PIA constraint. 
To me it sounds like SRT?}
%italic

Yes, indeed, both Nw and Dm are important parameters in describing PSDs. Our intent here is to draw the reader's attention to the
fact that, irrespective of the mathematical tools used, properly accounting for the PSD variability and its impact on observations
is a key issue. 

Regarding the PIA constraint, in principle, it is possible to independently derive the PIA from coincident radiometer observations.
From this perspective, the Surface Reference Technique is not the only means to derive the PIA estimates.  This is the reason why
we have not used the term SRT in the text.}

\item {\textit{to maximize the agreement between radiometer 
measurements .....actual radiometer measurements}\\
We rephrased the statement as suggested.  Regarding the agreement between simulation and observations, we have added the clarification
that the agreement is achieved by minimizing the differences between simulations and observations.}

\item {\textit{what does 'consistence' mean in this context?}\\
Consistency here means that the radiometer measurements simulated from the radar estimates are in agreement with their observed values. We added a statement in 
the text to clarify this.}

\item{\textit{To my understanding M\&P merged experimental DSDs within range of sequential rain rates...}\\
Thank you for your insight on this.
}

\item{\textit{the Rayleigh regime typically has the factor sqrt(epslon) in it.}\\
For simplicity and consistency with the definition of the size parameter in general text books on electromagnetic scattering, we have
omitted the factor $\sqrt{\epsilon}$ in the definition.}

\item {\textit{Eq 1.1 and the text following it.}\\
    We agree with the reviewer that the discussion may be unnecessary and even confusing.  Nevertheless, we thought it may be 
    useful to readers that may not be aware of the fact that $Z=\int _0 ^{\infty} N(D) D^6 dD$ holds only for the Rayleigh regime. 
    Unfortunately, there are textbooks that define $Z$ as the integral of $N(D) D^6 dD$ without any mention of the Rayleigh regime.
}
\item {\textit{This  procedure is termed ...}\\
Thank you for your insight on this.  We have included statements reflecting the reviewer's perspective and a reference 
in the text.
}

\item {\textit{aproximate.}\\
We have included the suggestion in the text.
}

\item {\textit{This property applies to any variable that is normalized by Nw.}\\
Thank you for the suggestion.  We have included it in the text.
}

\item{\textit{it seems that beta is constant exponent based on scattering simulations using a rain DSD model
How sensitive is beta to changes in water T, DSD shape changes or when the beam encounters
rain mixed with melting particles?}\\
Parameter $\beta$ does not strongly depend on the water temperature or the DSD shape.  However, it is sensitive to the 
phase of the particle. Strategies to account for the phase dependence of $\beta$ are discussed by Iguchi and Meneghini (1994)
and Grecu et al. (2011). We included a clarifying statement in the text.}

\item{\textit{that can be mitigated}\\
Thank you for noticing the incorrect formulation.  We have corrected it in the text.
}

\item{\textit{please improve figure quality eg by adding a panel that gives the differences}\\
Thank you for the suggestion. The differences among the simulated Ka-band reflectivity observations are meant to be rather qualitative and
give an idea of the impact of NUBF and multiple scattering on the simulations. For simplicity, we prefer to keep the figure as it is.
}

\item{\textit{Figure labeling.}\\
We apologize for the figure labeling errors. We believe they are caused by the Latex supporting files that were provided by the publisher. 
We refer to figures in the text using labels rather than numbers, and rely on the Latex compiler to assign the correct numbers to the figures.
The style file provided by the publisher does not seem to work properly on our system, but we believe the professional editor can easily fix the
figure numbering problem during the editorial process.}

\end{enumerate}

\end{document}